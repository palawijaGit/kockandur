\documentclass[a4paper,10pt]{scrbook}
\usepackage{makeidx}
\usepackage{graphicx}
\usepackage[magyar]{babel}
\usepackage[utf8]{inputenc}  

\begin{document}

\author{Mátyás-Peti Csaba}
\title{Mesterséges Intelligencia}
\date{2009 augusztus - }

\frontmatter
\tableofcontents

\mainmatter 

\chapter{Intelligens Ágensek}

\section{Alapfogalmak}

\textbf{\emph{Ágens (agent)}}: Bármit tekinthetünk ágensnek, ami képes arra hogy érzékelőivel 
(sensors) érzékelje a környezetét (enviroment) és beavatkozóival (actuators)
megváltoztassa azt. 

\bigskip
\noindent
\textbf{\emph{Érzékelés (percept)}}: Az ágens érzékelő bemeneteinek leírása egy 
tetszőleges pillanatban

\bigskip
\noindent
\textbf{\emph{Érzékelési soroza (percept sequence)}}: Az ágens érzéleléseinek teljes 
története, minden amit valaha is érzékelt. 

\bigskip
\noindent
\textbf{\emph{Ágensfüggvény (agent function)}}: Általában egy ágens viselkedését az adott 
pillanatban az addig megfigyelt teljes érzékelési sorozat határozza meg. Ezt 
matematikailag úgy fogalmazhatjuk meg, hogy veszünk egy \emph{ágensfüggvényt} mely 
érzékelési sorozatokat cselekvésékre képez le. 

\bigskip
\noindent 
\textbf{\emph{Ágensprogram (agent program)}}: Az ágensfüggvény egy absztrakt matematikai leírás 
(példuál egy táblázat). Az ehhez tartozó implementáció az ágens belsejében az ágensprogram.

\section{Jó viselkedés: A racionalítás koncepciója}

Első megközelítésben azt mondhatjuk hogy egy \textbf{\emph{racionális ágens (rational agent)}} 
olyan, mely helyesen cselekszik. Helyes cselekvésnek azt tekintjük ami sikeresebbé teszi. 
Ennek meghatározásához szükség van a sikeresség definiciójára. 

\subsection{Teljesítménymérték}

A teljesítményértéket jobb aszerint megállapítani hogy mit szeretnénk elérni a környezeten és nem 
az alapján hogy hogy szeretnénk hogy az ágens viselkedjen. 

\bigskip
\noindent
Pl ha egy porszívó ágens teljesítménymértékét feltakarított por mennyisége adja racionális viselkedés
lenne kiönteni a port és újra feltakarítani. 

\subsection{Racionalitás}

Racionalítást meghatározó tényezők:

\begin{itemize}
\item{Siker fokát mérő teljesítményérték}
\item{Ágens tudása a környezetéről}
\item{Környezeten végrehajtható cselekvések}
\item{Érzékelési sorozat az adott pillanatig} 
\end{itemize}

\end{document}
