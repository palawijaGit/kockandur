\documentclass[a4paper,12pt]{book}
\usepackage[magyar]{babel}
\usepackage[utf8]{inputenc} 
\begin{document}

Szabályozástechnika jegyzet - készült ``Szabályozástechnika 55079'' alapján

\chapter{Bevezetés}

Az írányitás egy folyamatba való beavatkozás egy cél elérésének az érdekében. Tipikus irányítási feladatok például egy szoba 
állandó hőmérsékleten tartása, autó sebbeségének irányított növelése illetve csökkentése. 

\section{Alapfogalmak}

\subsection*{Irányítás}
Az irányítás olyan művelet, amely valamely folyamatban annak elindítása, fenntartása, megfelelő lefolyásának biztosítása, megváltoztatása 
vagy megállítása végett beavatkozik. 

\subsection*{Kézi vagy automatikus irányítás}
Kézi irányítás esetén a megfigyelt mennyiség alapján a kezelő személy hoz döntést és avatkozik be a folyamatba. 

\noindent Automatikus irányítás esetén a döntéshozást gépi berendezések veszik át. 

\subsection*{Irányítási folyamatok alapműveletei}
\begin{itemize} 
\item{\emph{Érzékelés}: információszerzés a folyamatról és környezetről}
\item{\emph{Ítéletalkotás}: információk és cél alapján döntés a teendőkről }
\item{\emph{Rendelkezés}: utasítás a beavatkozásról }
\item{\emph{Jelformálás}: a beavatkozás módjának és jellegének befolyásolása}
\item{\emph{Végrehajtás}}
\end{itemize}

\subsection*{A jel fogalma, a jelek felosztása}

A jel az irányítási rendszerben megjelenő információ. Információszerzésre, továbbításra és tárolásra alkalmas. 
Van fizikai megjelenési formája (áram, feszültség, stb.) és van információtartalma, amelyet a hatás mutatja. 

\noindent Jelek különböző felosztásai 
\begin{itemize}
\item{Idő beli lefolyás szerint:}

\emph{Folyamatos}

\emph{Diszkrét idejű vagy mintavételezett:}

\item{Értékkészlet szerint}
 
\emph{Folytonos}

\emph{Szakaszos}

\item{Megjelenési forma alapján:}

\emph{Analóg}

\emph{Digitális}

\item{Érték meghatározottsága szerint}

\emph{Determinisztikus}

\emph{Sztohasztikus}

\end{itemize}

\subsection*{A rendszertechnikai összefüggések ábrázolása}

\emph{Szerkezeti vázlat: } Berendezésekről és azok kapcsolatáról ad áttekintést a rendszer irányítástechnikai szempontból
lényeges részeit feltüntetve. 

\emph{Működési vázlat: } A szerkezetek kapcsolódását és egymásra gyakorolt hatását mutatja be. Általában a szerkezeteket 
téglalapok szimbolizálják, egymásra gyakorolt hatásukat irányított (nyillal ellátott) vonalak jelzik. 

\emph{Hatásvázlat:} A ki- és bemenőjelek összefüggéseit adjuk meg. 

\subsection*{Vezérlés, szabályozás, zavarkompenzáció}

\emph{Vezérlés:} Az információt nem közvetlen az irányított jellemző (kimenőjel) mérésével nyerjük. 

\emph{Szabályozás:} Szabályozás esetén az irányítandóó mennyiséget (jellemzőt) érzékeljük. A szabályozási kör szervei: 
\emph{érzékelő szerv}, \emph{alapjelképző szerv}, \emph{különbségképző szerv}, \emph{erősítő és jelformáló szerv} valamit 
a \emph{végrehajtó és beavatkozó szerv}. 

Követelmények az érzékelővel kapcsolatban 
\begin{itemize}
\item{Megbizhatóan működjön a kivánt mérési tartományban}
\item{A tartományban legyen lineáris}
\item{Legyen pontos}
\item{Késleltetése legyen alacsony a folyamat időállandójához}
\item{Lehető legkisebb legyen a mérési zaj}
\end{itemize}


\chapter{Második fejezet}

Powered by \LaTeX

\end{document}
