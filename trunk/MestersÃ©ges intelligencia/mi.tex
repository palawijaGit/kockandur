\documentclass[a4paper,10pt]{scrbook}
\usepackage[magyar]{babel}
\usepackage[utf8]{inputenc} 

\begin{document}

Mesterséges Intelligencia 

\chapter{Intelligens Ágensek}

\section{Alapfogalmak}

\emph{Ágens (agent)}: Bármit tekinthetünk ágensnek, ami képes arra hogy érzékelőivel 
(sensors) érzékelje a környezetét (enviroment) és beavatkozóival (actuators)
megváltoztassa azt. 

\bigskip
\noindent
\emph{Érzékelés (percept)}: Az ágens érzékelő bemeneteinek leírása egy 
tetszőleges pillanatban

\bigskip
\noindent
\emph{Érzékelési soroza (percept sequence)}: Az ágens érzéleléseinek teljes 
története, minden amit valaha is érzékelt. 

\bigskip
\noindent
\emph{Ágensfüggvény (agent function)}: Általában egy ágens viselkedését az adott 
pillanatban az addig megfigyelt teljes érzékelési sorozat határozza meg. Ezt 
matematikailag úgy fogalmazhatjuk meg, hogy veszünk egy \emph{ágensfüggvényt} mely 
érzékelési sorozatokat cselekvésékre képez le. 

\bigskip
\noindent 
\emph{Ágensprogram (agent program)}: Az ágensfüggvény egy absztrakt matematikai leírás 
(példuál egy táblázat). Az ehhez tartozó implementáció az ágens belsejében az ágensprogram.

\section{Jó viselkedés: A racionalítás koncepciója}

Első megközelítésben azt mondhatjuk hogy egy \emph{racionális ágens (rational agent)} 
olyan, mely helyesen cselekszik. Helyes cselekvésnek azt tekintjük ami sikeresebbé teszi. 
Ennek meghatározásához szükség van a sikeresség definiciójára. 

\subsection{Teljesítménymérték}


\end{document}
